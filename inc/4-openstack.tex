\section[ЛР №4. Модель обслуживания IaaS, OpenStack]{Лабораторная работа №4. \\
Модель обслуживания IaaS на примере создания инстанса в OpenStack}

\textbf{Цель работы:} ознакомиться с веб-интерфейсом OpenStack, научиться настраивать сеть, создавать инстансы и управлять ими в OpenStack.

\subsection{Краткие сведения об OpenStack}

Ранее в лабораторной работе №3, мы уже использовали готовые PaaS-решения от облачных провайдеров, однако если же мы хотим использовать более гибкие решения или же самим стать облачным провайдером, то необходимо использование IaaS-решений, таких как OpenStack.

С помощью OpenStack можно создавать платформы облачных вычислений для частных и публичных облаков.
OpenStack был начат как совместный проект между Rackspace и NASA в 2010 году.
С 2012 года им управляет некоммерческая организация OpenStack Foundation.

Теперь OpenStack поддерживают более чем 500 сторонних организаций.
OpenStack является открытым проектом, исходные коды распространяются под лицензией Apache 2.0.

Помимо обеспечения IaaS-решений, OpenStack развивалась с течением времени, чтобы предоставлять другие услуги, как базы данных, системы хранения данных и прочее.

Благодаря модульной архитектуре OpenStack, любой желающий может добавить дополнительные компоненты, чтобы получить специфические особенности или функциональные возможности.

Некоторые из основных компонентов OpenStack:
\begin{itemize}
    \item Keystone, инструментом идентификации пользователей;
    \item Nova, диспетчер облачного процесса вычислений;
    \item Horizon, панель управления для OpenStack;
    \item Neutron, реализация сети в качестве сервиса, обеспечение сетевых возможностей для различных компонентов;
    \item Glance, используется для управления образами ОС, которые требуются для работающих экземпляров (инстансов);
    \item Swift, распределенное хранилище высокой доступности;
    \item Cinder, блочное хранилище;
    \item Heat, инструмент оркестровки, позволяет запускать готовые облачные архитектуры из шаблонов, описанных текстом;
    \item Celiometer, инструмент для сбора различных статистических данных в облаке.
\end{itemize}

Каждый из компонентов OpenStack является также модульным.
Например, с помощью Nova мы можем выбрать гипервизор в зависимости от требований, Libvirt (QEMU/KVM), Hyper-V, VMware, XenServer, Xen с Libvirt.

Преимущества использования OpenStack:
\begin{itemize}
    \item решение с открытым исходным кодом;
    \item платформа облачных вычислений для публичных и частных облаков;
    \item предлагает гибкое и настраиваемое окружение;
    \item обеспечивает высокий уровень безопасности;
    \item облегчает автоматизацию на протяжении всех этапов жизненного цикла облака;
    \item за счет снижения затрат на управление системой и привязанности к вендору, это может быть экономически эффективным.
\end{itemize}

\subsection{Порядок выполнения работы}

Для выполнения данной лабораторной работы необходим аккаунт на Facebook.

Пример работы с TryStack представлен в прил.~\ref{pril:f}.

\begin{enumerate}
    \item Вступить в группу TryStack на Facebook;
    \item Авторизоваться на сервисе TryStack;
    \item Ознакомиться с интерфейсом OpenStack;
    \item Согласно прил.~\ref{pril:f} настроить окружение для создания инстанса;
    \item Создать тестовый инстанс и разместить любое приложение (по желанию);
    \item Проверить работоспособность приложения в облаке.
\end{enumerate}

\subsection{Контрольные вопросы}
\begin{enumerate}
    \item В чем преимущество и недостатки использования IaaS перед PaaS?
    \item Назовите несколько примеров облачных поставщиков IaaS?
    \item Почему использование SSH-ключей безопаснее чем парольная аутентификация?
    \item В чем преимущество модульной архитектуры перед монолитной?
\end{enumerate}

\clearpage
